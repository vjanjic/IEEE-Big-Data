Future-generation healthcare centers will be highly distributed, combining centralised hospital systems with home-, work- and environment-based monitoring and diagnostics systems in order to reduce costs and injury-related risks while at the same time improving the quality of service that the patients receive and reducing the response time for diagnostics and treatment. This will require medical data will to be accessed and shared over a variety of mediums, including untrusted networks. In this paper, we present a design and initial implementation of the \emph{Serums} toolchain for accessing, storing, communicating and analysing highly confidential medical data in a safe, secure and privacy-preserving way. In addition, we describe a \emph{data fabrication} framework for generating large volumes of synthetic but realistic data, that is used in the process of design and evaluation of the toolchain. We demonstrate the initial versions of the \emph{Serums} techniques on a Edinburgh Cancer Gateway use case from National Health Service (NHS) Scotland, where information about the effects of treatments of cancer patients is collected from different distributed databases and analysed to adapt the treatment.

Abstract due: 12 August 2019

Paper due: 30 August 2019

\url{http://bigdataieee.org/BigData2019/}
