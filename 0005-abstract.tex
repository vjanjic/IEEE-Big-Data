Future-generation healthcare systems will be highly distributed, combining centralised hospital systems with decentralised home-, work- and environment-based monitoring and diagnostics systems. These will reduce costs and injury-related risks whilst both improving quality of service, and reducing the response time for diagnostics and treatments made available to patients. To make this vision possible, medical data must be accessed and shared over a variety of mediums including untrusted networks. In this paper, we present the design and initial implementation of the \emph{SERUMS} tool-chain for accessing, storing, communicating and analysing highly confidential medical data in a safe, secure and privacy-preserving way. In addition, we describe a data fabrication framework for generating large volumes of synthetic but realistic data, that is used in the design and evaluation of the tool-chain. We demonstrate the present version of our technique on a use case derived from the Edinburgh Cancer Centre, NHS Lothian, where information about the effects of chemotherapy treatments on cancer patients is collected from different distributed databases, analysed and adapted to improve ongoing treatments.

%Abstract due: 10 August 2019

%Paper due: 19 August 2019

%\url{http://bigdataieee.org/BigData2019/}
