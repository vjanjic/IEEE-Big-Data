\noindent
%The health centers of the future will be highly distributed and, due to convenience and cost effectiveness, will need to integrate home-, work- and environment-based monitoring with centralised hospital diagnostics and treatment services. 
In this paper, we have outlined the problems that the distributed health systems of the future will face in terms of safe storing and sharing of confidential patient data. %, especially taking into account more and more strict regulations and legislations (such as GDPR) that govern the ownership and access rights to personal data. 
We have also proposed the \emph{SERUMS} methodology for managing confidential, distributed medical data, covering all the phases in its lifetime, from retrieval and storing to end-point data analytics. Furthermore, we have described the initial versions of the tools from the \emph{SERUMS} tool-chain, including new universal smart patient record format, blockchain for controlling access to the health records and recording lineage of the data, authentication mechanisms for logging in to healthcare systems and privacy-preserving data analytics techniques. We have also described Data Fabrication Platform (DFP), a platform for generating large volumes of synthetic but realistic medical data that will be used for development and evaluation of the \emph{SERUMS} tool-chain. Finally, we have described its proposed use in the \emph{Edinburgh Cancer Gateway} use case that collects and analyses information about effects of chemotherapy treatments on breast cancer patients, to predict the outcome of the treatment and improve treatment.
%
%In future, we plan to develop the final versions of the tools in the \emph{SERUMS} tool-chain, investigate interoperability issues between them and integrate them into the \emph{SERUMS} methodology. We also plan to evaluate the tool-chain on two further large-scale use cases - a system for a new hospital that will be built by Zuyderland Medisch Centrum, one of the \emph{SERUMS} project partners, and a system for Fundacio Clinic per a Reserca Biomedica in Barcelona.
